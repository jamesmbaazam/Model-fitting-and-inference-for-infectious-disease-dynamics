\documentclass[paper=a4, 11pt]{scrlttr2}
\usepackage[english]{babel}
\usepackage{graphicx}
\usepackage{palatino}
\KOMAoptions{fromlogo=true, addrfield=false, backaddress=off, foldmarks=off}
\makeatletter
\@setplength{firstheadhpos}{23mm}
\@setplength{toaddrhpos}{23mm}
\@setplength{toaddrvpos}{43mm}
\@setplength{refvpos}{40mm}
\@setplength{toaddrwidth}{200mm}
\@setplength{sigbeforevskip}{0mm}
\makeatother
\setkomavar{date}{June 5, 2015}
\setkomavar{signature}{Seb \& Anton}
\setkomavar{fromlogo}{\includegraphics[%
           width=165mm,clip]%
           {/Users/seb/Research/Docs/lshtmletterhead.pdf}}
\setkomavar{firstfoot}{\hrule \vspace{0.2cm} {\bf Improving health worldwide}}
 \setkomavar{firsthead}{\usekomavar{fromlogo} \vspace{0.5cm} \\%
     \usekomavar{fromname} \\\usekomavar{fromemail} \\\usekomavar{fromphone}}
\usepackage[top=14mm, bottom=20mm, left=23mm, right=23mm]{geometry}
\usepackage{hyperref}
\hypersetup{pdfpagemode=UseNone} % don't show bookmarks on initial view
\hypersetup{colorlinks, urlcolor={blue}}

\begin{document}
\begin{letter}{}
\opening{Dear participant,}
Thanks for registering to our short course on ``Model fitting and
inference for infectious disease dynamics'', 16--19 June at the London
School of Hygiene \& Tropical Medicine (LSHTM). A few notes ahead of
the course:

\begin{enumerate}
\item Remember that you are expected to bring your own laptop with R
installed. Please make sure you have a recent version of R installed,
ideally the most recent one (3.2.0). If any of that is a problem, let us know
and we will come up with a solution. We will also have some replacement laptops
available, in case something breaks during the course.
\item An internet connection will be required for the course. The easiest
way for most of you is probably to connect through \emph{eduroam} --- if you
haven't got this set up on your laptop it might be worth checking how you
can get access. If you are not based at an academic institution or can't
get access to eduroam for another reason, we will set you up to connect
through LSHTM's own wireless network.
\item You will find attached an introduction to the R commands and concepts
we will use during the course. If you don't use R in your daily work, it
would be useful for you to work through this ahead of the course. Even
if you are a regular R user, it might be worth having a quick read to
make sure we will be on the same page.
\item We will use the R package \texttt{pomp} in one of the
  sessions. To save time, it would be great if you could try and
  install it using the instructions provided at the 
  \href{http://pomp.r-forge.r-project.org/vignettes/getting_started.html#installing-the-package}{package web site},
  paying particular attention to the ``Important information for
  Windows and Mac users.'' Again, let us know if you encounter any
  problems with this.
\item Also attached is a timetable. We will start at 10am on Tuesday 16th,
and close at 3pm on Friday 19th. Lunches are not included in the
course fee. There are plenty of very reasonbly priced lunch options in
and around the London School --- we will give you more information on
this during the course.
\item We have booked a nearby restaurant serving simple English cuisine for
Thursday, 16th June, 7pm. We are hoping that as many as possible of you
can join --- if you already know that you won't be able to come to this,
please let us know to facilitate planning. The price of the dinner will
be covered by the course fee. There will be vegetarian options --- if you
have any other dietary requirements, please let us know as soon as
possible.
\item We are hoping for in interesting and stimulating course. If there are
any burning issues on your mind that you would like discussed, please
let us know and we'll try our best to accommodate your wishes.
\item Any further questions, please do not hesitate to email us at
\href{mailto:mfiidd.lshtm@gmail.com}{mfiidd.lshtm@gmail.com}.
\end{enumerate}

\closing{See you soon in London,}
\end{letter}
\end{document}
